\documentclass[english,notitlepage]{revtex4-1}  % defines the basic parameters of the document
%For preview: skriv i terminal: latexmk -pdf -pvc filnavn



% if you want a single-column, remove reprint

% allows special characters (including æøå)
\usepackage[utf8]{inputenc}
%\usepackage[english]{babel}

%% note that you may need to download some of these packages manually, it depends on your setup.
%% I recommend downloading TeXMaker, because it includes a large library of the most common packages.

\usepackage{physics,amssymb}  % mathematical symbols (physics imports amsmath)
\usepackage{graphicx}         % include graphics such as plots
\usepackage{xcolor}           % set colors
\usepackage{hyperref}         % automagic cross-referencing (this is GODLIKE)
\usepackage{tikz}             % draw figures manually
\usepackage{listings}         % display code
\usepackage{subfigure}        % imports a lot of cool and useful figure commands
%\usepackage{float}

% defines the color of hyperref objects
% Blending two colors:  blue!80!black  =  80% blue and 20% black
\hypersetup{ % this is just my personal choice, feel free to change things
    colorlinks,
    linkcolor={red!50!black},
    citecolor={blue!50!black},
    urlcolor={blue!80!black}}

%% Defines the style of the programming listing
%% This is actually my personal template, go ahead and change stuff if you want
\lstset{ %
	inputpath=,
	backgroundcolor=\color{white!88!black},
	basicstyle={\ttfamily\scriptsize},
	commentstyle=\color{magenta},
	language=Python,
	morekeywords={True,False},
	tabsize=4,
	stringstyle=\color{green!55!black},
	frame=single,
	keywordstyle=\color{blue},
	showstringspaces=false,
	columns=fullflexible,
	keepspaces=true}


%% USEFUL LINKS:
%%
%%   UiO LaTeX guides:        https://www.mn.uio.no/ifi/tjenester/it/hjelp/latex/
%%   mathematics:             https://en.wikibooks.org/wiki/LaTeX/Mathematics

%%   PHYSICS !                https://mirror.hmc.edu/ctan/macros/latex/contrib/physics/physics.pdf

%%   the basics of Tikz:       https://en.wikibooks.org/wiki/LaTeX/PGF/TikZ
%%   all the colors!:          https://en.wikibooks.org/wiki/LaTeX/Colors
%%   how to draw tables:       https://en.wikibooks.org/wiki/LaTeX/Tables
%%   code listing styles:      https://en.wikibooks.org/wiki/LaTeX/Source_Code_Listings
%%   \includegraphics          https://en.wikibooks.org/wiki/LaTeX/Importing_Graphics
%%   learn more about figures  https://en.wikibooks.org/wiki/LaTeX/Floats,_Figures_and_Captions
%%   automagic bibliography:   https://en.wikibooks.org/wiki/LaTeX/Bibliography_Management  (this one is kinda difficult the first time)
%%   REVTeX Guide:             http://www.physics.csbsju.edu/370/papers/Journal_Style_Manuals/auguide4-1.pdf
%%
%%   (this document is of class "revtex4-1", the REVTeX Guide explains how the class works)


%% CREATING THE .pdf FILE USING LINUX IN THE TERMINAL
%%
%% [terminal]$ pdflatex template.tex
%%
%% Run the command twice, always.
%% If you want to use \footnote, you need to run these commands (IN THIS SPECIFIC ORDER)
%%
%% [terminal]$ pdflatex template.tex
%% [terminal]$ bibtex template
%% [terminal]$ pdflatex template.tex
%% [terminal]$ pdflatex template.tex
%%
%% Don't ask me why, I don't know.

\begin{document}
\title{Project 1 - FYS3150}      % self-explanatory
\author{René Ask and Benedicte Nyheim}          % self-explanatory
\date{\today}                             % self-explanatory
\noaffiliation                            % ignore this
                                          % marks the end of the abstracthttps://github.com/reneaas/fys2160.git
\maketitle                                % creates the title, author, date & abstract

\section{Method}
We're going to solve the differential equation 
\begin{equation}\label{diff_eq}
	-u''(x) = f(x), \quad x \in (0,1), \quad u(0)=u(1)=0.
\end{equation}
We'll approximate this differential equation by a function $v(x) \approx u(x)$ by the 
approximation scheme 
\begin{equation}
	-\frac{-v_{i+1}+v_{i-1} - 2v_i}{h^2} = f_i, \quad i=1,2,...,n,
\end{equation}
which may be rearranged into 
\begin{equation}\label{approx_1}
	2v_i - v_{i+1} - v_{i-1} = f_ih^2 \equiv \tilde{b_i}.
\end{equation}
From \eqref{approx_1} we can write 
\begin{equation}
	\begin{pmatrix}
	2v_1 - v_2 \\ 
	-v_1 + 2v_2 - v_3 \\ 
	\vdots \\
	2v_n - v_{n-1}
	\end{pmatrix}
	=
	\begin{pmatrix}
	2 & -1 & 0  & \cdots & 0 \\
	-1 & 2 & -1 & \cdots & 0 \\
	\vdots \\
	0 & \cdots & 0 & -1 & 2 
	\end{pmatrix}
	\begin{pmatrix}
	v_1 \\ v_2 \\ \vdots \\ v_n
	\end{pmatrix}
	= \begin{pmatrix}
	\tilde{b_1} \\ \tilde{b_2} \\ \vdots \\ \tilde{b_n}
	\end{pmatrix}
\end{equation}

To this end we will develop an algorithm based on the LU-decomposition $A = LU$:
\begin{equation} A= 
	\begin{pmatrix}
	b_1 & c_1 & 0 & \cdots & \cdots & \cdots \\
	a_1 & b_2 & c_2 & 0  &\cdots & \cdots  \\ 
	0 & a_2 & b_3 & 0 & \cdots & \cdots \\
	\vdots & \vdots & \vdots & \vdots & \ddots & \vdots \\
	0 & 0 & \cdots & a_{n-2} & b_{n-1} & c_{n-1} \\
	0 & 0 & \cdots  & 0 & a_{n-1} & b_n
	\end{pmatrix}
	= 
	\begin{pmatrix}
	1 & 0 & \cdots &  \cdots & \cdots & 0 \\
	\ell_2 & 1 & \cdots & \cdots & \cdots & 0 \\
	0 & \ell_3 & 1 & \cdots & \cdots & 0 \\
	\vdots & \vdots & \vdots & \ddots & \vdots \\
	0 & 0 & \cdots & \ell_{n-1} & 1 & 0 \\
	0 & 0 & \cdots &\cdots & \ell_n & 1  
	\end{pmatrix}
	\begin{pmatrix}
	d_1 & u_1 & \cdots & \cdots &\cdots & 0 \\ 
	0 & d_2 & u_2 & \cdots & \cdots & 0 \\
	0 & 0 & d_3 & u_3 & \cdots & 0 \\
	\vdots & \vdots & \vdots & \ddots & \vdots & \vdots \\
	0 & 0 & \cdots & \cdots & d_{n-1} & u_{n-1} \\
	0 & 0 & \cdots & \cdots & 0 & d_n
	\end{pmatrix}
	= LU,
\end{equation}
and performing matrix multiplication we get 
\begin{equation}
	LU = 
	\begin{pmatrix}
	d_1 & u_1 & \cdots & \cdots &\cdots & 0 \\ 
	\ell_2d_1 & \ell_2u_1 + d_2 & u_2 & \cdots & \cdots & 0 \\
	0 & \ell_3d_2 & \ell_3u_2 + d_3 & u_3 & \cdots & 0 \\
	\vdots & \vdots & \vdots & \ddots & \vdots & \vdots \\
	0 & 0 & \cdots & \ell_{n-1}d_{n-2} & \ell_{n-1}u_{n-2} + d_{n-1} & u_{n-1} \\
	0 & 0 & \cdots & \cdots & \ell_nd_{n-1} & \ell_n u_{n-1} + d_n
	\end{pmatrix},
\end{equation}
which yields the following general relations: 
\begin{gather}
	b_i = d_i, \qquad c_i = u_i, \qquad \text{for} \qquad i = 1,\\
	\ell_i = \frac{a_{i-1}}{d_{i-1}}, \qquad \text{for} \qquad 1 < i < n, \\
	d_i = b_i - \ell_iu_{i-1}, \qquad \text{for} \qquad 1 < i < n, \\
	\ell_n = \frac{a_{n-1}}{d_{n-1}}, \qquad d_n = b_n - \ell_n u_{n-1}, \qquad \text{for} \qquad i = n.
\end{gather}

Now, assuming that $b_1 = b_2 = \cdots = b_n$ and $a_1 = c_1$, $a_2 = c_2,  \cdots, a_{n-1} = c_{n-1}$, we can simplify the relations as 
\begin{gather}
	
\end{gather}


\end{document}