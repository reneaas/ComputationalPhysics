\documentclass[english,notitlepage]{revtex4-1}  % defines the basic parameters of the document
%For preview: skriv i terminal: latexmk -pdf -pvc filnavn



% if you want a single-column, remove reprint

% allows special characters (including æøå)
\usepackage[utf8]{inputenc}
%\usepackage[english]{babel}

%% note that you may need to download some of these packages manually, it depends on your setup.
%% I recommend downloading TeXMaker, because it includes a large library of the most common packages.

\usepackage{physics,amssymb}  % mathematical symbols (physics imports amsmath)
\usepackage{graphicx}         % include graphics such as plots
\usepackage{xcolor}           % set colors
\usepackage{hyperref}         % automagic cross-referencing (this is GODLIKE)
\usepackage{tikz}             % draw figures manually
\usepackage{listings}         % display code
\usepackage{subfigure}        % imports a lot of cool and useful figure commands
%\usepackage{float}

% defines the color of hyperref objects
% Blending two colors:  blue!80!black  =  80% blue and 20% black
\hypersetup{ % this is just my personal choice, feel free to change things
    colorlinks,
    linkcolor={red!50!black},
    citecolor={blue!50!black},
    urlcolor={blue!80!black}}

%% Defines the style of the programming listing
%% This is actually my personal template, go ahead and change stuff if you want
\lstset{ %
	inputpath=,
	backgroundcolor=\color{white!88!black},
	basicstyle={\ttfamily\scriptsize},
	commentstyle=\color{magenta},
	language=Python,
	morekeywords={True,False},
	tabsize=4,
	stringstyle=\color{green!55!black},
	frame=single,
	keywordstyle=\color{blue},
	showstringspaces=false,
	columns=fullflexible,
	keepspaces=true}


%% USEFUL LINKS:
%%
%%   UiO LaTeX guides:        https://www.mn.uio.no/ifi/tjenester/it/hjelp/latex/
%%   mathematics:             https://en.wikibooks.org/wiki/LaTeX/Mathematics

%%   PHYSICS !                https://mirror.hmc.edu/ctan/macros/latex/contrib/physics/physics.pdf

%%   the basics of Tikz:       https://en.wikibooks.org/wiki/LaTeX/PGF/TikZ
%%   all the colors!:          https://en.wikibooks.org/wiki/LaTeX/Colors
%%   how to draw tables:       https://en.wikibooks.org/wiki/LaTeX/Tables
%%   code listing styles:      https://en.wikibooks.org/wiki/LaTeX/Source_Code_Listings
%%   \includegraphics          https://en.wikibooks.org/wiki/LaTeX/Importing_Graphics
%%   learn more about figures  https://en.wikibooks.org/wiki/LaTeX/Floats,_Figures_and_Captions
%%   automagic bibliography:   https://en.wikibooks.org/wiki/LaTeX/Bibliography_Management  (this one is kinda difficult the first time)
%%   REVTeX Guide:             http://www.physics.csbsju.edu/370/papers/Journal_Style_Manuals/auguide4-1.pdf
%%
%%   (this document is of class "revtex4-1", the REVTeX Guide explains how the class works)


%% CREATING THE .pdf FILE USING LINUX IN THE TERMINAL
%%
%% [terminal]$ pdflatex template.tex
%%
%% Run the command twice, always.
%% If you want to use \footnote, you need to run these commands (IN THIS SPECIFIC ORDER)
%%
%% [terminal]$ pdflatex template.tex
%% [terminal]$ bibtex template
%% [terminal]$ pdflatex template.tex
%% [terminal]$ pdflatex template.tex
%%
%% Don't ask me why, I don't know.

\begin{document}
\title{Project 1 - FYS3150}      % self-explanatory
\author{René Ask}          % self-explanatory
\date{\today}                             % self-explanatory
\noaffiliation                            % ignore this
                                          % marks the end of the abstracthttps://github.com/reneaas/fys2160.git
\maketitle                                % creates the title, author, date & abstract

\section{Theory}
We're going to solve the differential equation 
\begin{equation}\label{diff_eq}
	-u''(x) = f(x), \quad x \in (0,1), \quad u(0)=u(1)=0.
\end{equation}
We'll approximate this differential equation by a function $v(x) \approx u(x)$ by the 
approximation scheme 
\begin{equation}
	-\frac{-v_{i+1}+v_{i-1} - 2v_i}{h^2} = f_i, \quad i=1,2,...,n,
\end{equation}
which may be rearranged into 
\begin{equation}\label{approx_1}
	2v_i - v_{i+1} - v_{i-1} = f_ih^2 \equiv b_i.
\end{equation}
From \eqref{approx_1} we can write 
\begin{equation}
	\begin{pmatrix}
	2v_1 - v_2 \\ 
	-v_1 + 2v_2 - v_3 \\ 
	\vdots \\
	2v_n - v_{n-1}
	\end{pmatrix}
	=
	\begin{pmatrix}
	2 & -1 & 0  & \cdots & 0 \\
	-1 & 2 & -1 & \cdots & 0 \\
	\vdots \\
	0 & \cdots & 0 & -1 & 2 
	\end{pmatrix}
	\begin{pmatrix}
	v_1 \\ v_2 \\ \vdots \\ v_n
	\end{pmatrix}
	= \begin{pmatrix}
	b_1 \\ b_2 \\ \vdots \\ b_n
	\end{pmatrix}
\end{equation}

To this end we want to develop a general algorithm to solve the equation above. Suppose we've got a tri-diagonal matrix $A$ and want to decompose it using LU-decomposition such that $A = LU$ where $L$ is a lower-triangular matrix with ones on its diagonal and $U$ is an upper-triangular matrix. Suppose that $A,L,U \in \mathbb{R}^{n\times n}$ such that 
\begin{equation} A= 
	\begin{pmatrix}
	b_1 & c_1 & 0 & \cdots & \cdots & \cdots \\
	a_1 & b_2 & c_2 & 0  &\cdots & \cdots  \\ 
	0 & a_2 & b_3 & 0 & \cdots & \cdots \\
	\vdots & \vdots & \vdots & \vdots & \ddots & \vdots \\
	0 & 0 & \cdots & a_{n-2} & b_{n-1} & c_{n-1} \\
	0 & 0 & \cdots  & 0 & a_{n-1} & b_n
	\end{pmatrix}
	= 
	\begin{pmatrix}
	1 & 0 & \cdots & 0\\
	l_{21} & 1 &  \cdots & 0 \\ 
	\vdots  & \ddots & \vdots & \vdots  \\
	l_{n1} & l_{n2} & \cdots & 1  
	\end{pmatrix}
	\begin{pmatrix}
	u_{11} & u_{12} & \cdots & u_{1n} \\ 
	u_{21} & u_{22} & \cdots & u_{2n} \\
	\vdots & \vdots & \ddots & \vdots \\
	u_{n1} & u_{n2} & \cdots & u_{nn}
	\end{pmatrix}
	= LU.
\end{equation}
Performing matrix multiplication yields 
\begin{equation}\label{LU_decomp}
	LU = \begin{pmatrix}
	u_{11} & u_{12} & \cdots & u_{1n} \\
	l_{21}u_{11} + u_{21} & l_{21}u_{12} + u_{22} & \cdots & l_{21}u_{1n} + u_{2n} \\
	\vdots & \vdots & \ddots & \vdots \\
	l_{n1}u_{11} + l_{n2}u_{21} + \cdots + u_{n1} & l_{n1}u_{12} + l_{n2}u_{22} + \cdots + u_{n2} & \cdots & l_{n1}u_{1n} + l_{n2}u_{2n} + \cdots + u_{nn}
	\end{pmatrix},
\end{equation}
thus each matrix element of $A$ which we'll denote $A_{ij}$ may be computed by the general formula 
\begin{equation}\label{matrix_elements}
	A_{ij} = u_{ji} + \sum_{k = 1}^{i-1} l_{ik}u_{kj},
\end{equation}
which yields in combination with \eqref{LU_decomp}
\begin{gather}
	b_i = A_{ii} \qquad \text{for} \qquad i = 1,2,...,n \\
	a_i = A_{i+1,i} \qquad \text{for} \qquad i = 1,2,...,n-1 \\
	c_i = A_{i, i+1} \qquad \text{for} \qquad i = 1,2,...,n-1\\ 
	A_{ij} = 0 \qquad \text{otherwise}\\
	A_{ij} = 0 \qquad \text{for} \qquad (i,j) \neq (i,i) \qquad  \text{or} \qquad (i,j) \neq (i,i+1) \qquad \text{or} \qquad (i,j) \neq (i+1,i)
\end{gather}



\end{document}